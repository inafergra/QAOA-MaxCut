
\section{Conclusions and future outlook}
In this project we have studied the Quantum Approximate Optimization Algorithm (QAOA) for the Max-Cut problem. We implemented the code from scratch and we run it for different classical simulators. We then made some modifications to the algorithm so that the number of evaluations to the objective function could be decreased, and finally we run it on two real quantum processors: Vigo from IBM Quantum Experience and Starmon 5 from QuTech. 

The performance of the algorithm in the most ideal case is promising, as we have seen in sections \ref{statevector_section} and \ref{qasm_section}, specially when we apply several layers of the algorithm. It quickly converges for the simplest graphs and gives a good approximation ration for more difficult graphs. But as we introduce noise, we observed that the errors in each layer quickly add up, and thus running the algorithm with many layers becomes counter-productive. For the real backends we have used, this optimal $p$ is 1, meaning that it is not efficient to run QAOA for $p>1$ in this devices. 

The results we have obtained, although very limited, point to a more general conclusion. The noise levels in current Quantum Computer only allow for QAOA to be run with a very low number of layers. For QAOA to present some advantage over classical algorithms, a high number of layers is likely to be needed \cite{Guerreschi_2019}. The future of this algorithm is therefore, like many other quantum algorithms, dependent on the improvement of the quantum hardware as many more qubits, with greater quality and connectivity, are needed. One of the reasons driving this interest in QAOA is the kind of problems it will solve. This kind of combinatorial optimization problems come up in many fields of industry, logistics, finance, science etc. Currently, QAOA is already being applied to some (very small) instances of real problems such as traffic control \cite{zhang2020qed} and flight scheduling \cite{Vikst_l_2020}. Still, the size of the analyzed data in this kind of papers is very small as the hardware is still veyr limited.

Nevertheless, the possible quantum advantage of QAOA over classical algorithms is not something that has been rigorously proven. In fact, M. Hastings published a paper in 2019 where a classical algorithm was mathematically shown to outperform QAOA for $p<1000$ \cite{hastings2019classical}. Therefore, although QAOA is thought to be very promising for the NISQ (Noisy Intermediate Scale Quantum) devices era, current research is casting some doubts on the subject. Despite this, there is a vast amount of scientific literature being produced about QAOA and, more generally, about Variational Quantum Circuits, making the future of this algorithms a hot topic within the quantum computing community. 

